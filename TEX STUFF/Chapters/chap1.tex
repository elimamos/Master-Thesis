\documentclass[twoside,a4paper,10]{book}
\usepackage{graphicx}
\usepackage{hyperref}
\usepackage{amsmath}
\usepackage{amssymb}
\usepackage{textcomp}
\usepackage[utf8]{inputenc}
\usepackage[polish]{babel}
\usepackage[T1]{fontenc}
\usepackage{standalone}
\usepackage{array}
% pakiet stosowany do url'i w bibliografii, zamienia odnośniki na ładnie sformatowane
\usepackage{url}


% pakiety służące do numerowania i tworzenia algorytmów
\usepackage{algorithmic}
\usepackage{algorithm}
% redefinicja etykiety nagłówkowej listy algorytmów, domyślna jest po angielsku
\renewcommand{\listalgorithmname}{Spis algorytmów}

\usepackage[section]{placeins}
\usepackage{pdfpages}

% pakiet do wyliczania skali, przydatny przy dużych obrazkach
\usepackage{pgf}
% pakiet służący do automatycznego sortowania odnośników do bibliografii
\usepackage[sort]{natbib}

% tworzenie listingów
\usepackage{listings}
% tworzenie figur wewnątrz figur
\usepackage{subfig}
% do automatycznego skracania nazw rozdziałów i podrozdziałów używanych w nagłówkach strony by mieściły się w jednej linii
\usepackage[fit]{truncate}
% fancyhdr - ładne nagłówki, definicja wyglądu nagłówka, numery stron będą umieszczane w nagłówku po odpowiedniej stronie
\usepackage{fancyhdr}
\pagestyle{fancy}
\renewcommand{\chaptermark}[1]{\markboth{#1}{}}
\renewcommand{\sectionmark}[1]{\markright{\thesection\ #1}}



\fancyhf{}
\fancyhead[LE,RO]{\bfseries\thepage}
% tutaj ograniczamy szerokość pola w nagłówku zawierającego nazwę rozdziału/podrozdziału do 95% szerokości strony
% redefinicja sposobu prezentacji nazw domyślnie wypisywanych wielkimi literami (np. domyślnie w nagłówku Spis treści będzie miał postać SPIS TREŚCI)
% Uwaga! to może popsuć wielkie litery w ogóle! Jak coś nie działa należy usunąć \nouppercase{} z poniższych definicji
\fancyhead[LO]{\nouppercase{\bfseries{\truncate{.95\headwidth}{\rightmark}}}}
\fancyhead[RE]{\nouppercase{\bfseries{\truncate{.95\headwidth}{\leftmark}}}}
\renewcommand{\headrulewidth}{0.5pt}
\renewcommand{\footrulewidth}{0pt}

% definicja typu prostego wymagana przez pierwsze strony rozdziałów itp.
% powyższe reguły niestety tych stron nie dotyczą, gdyż Latex automatycznie przełącza je pomiędzy fancy a plain
% w tym wypadku eliminujemy nagłówki i stopki na stronach początkowych
\fancypagestyle{plain}{%
 \fancyhead{}
 \fancyfoot{}
 \renewcommand{\headrulewidth}{0pt}
 \renewcommand{\footrulewidth}{0pt}
}

\parskip 0.05in


% makro umożliwiające otaczanie symboli okręgami
\usepackage{tikz}
% brak justowania tekstu (bazą okręgu będzie linia tekstu)
\newcommand*\mycirc[1]{
  \begin{tikzpicture}
    \node[draw,circle,inner sep=1pt] {#1};
  \end{tikzpicture}}

% pionowe justowanie tekstu, środek okręgu pokrywa się ze środkiem tekstu
\newcommand*\mycircalign[1]{%
  \begin{tikzpicture}[baseline=(C.base)]
    \node[draw,circle,inner sep=1pt](C) {#1};
  \end{tikzpicture}}

% zmiana nazwy twierdzeń i lematów
\newtheorem{theorem}{Twierdzenie}[section]
\newtheorem{lemma}[theorem]{Lemat}

% tworzenie definicji dowodu
\newenvironment{proof}[1][Dowód]{\begin{trivlist}
\item[\hskip \labelsep {\bfseries #1}]}{\end{trivlist}}
% \newenvironment{definition}[1][Definicja]{\begin{trivlist}
% \item[\hskip \labelsep {\bfseries #1}]}{\end{trivlist}}
% \newenvironment{example}[1][Przykład]{\begin{trivlist}
% \item[\hskip \labelsep {\bfseries #1}]}{\end{trivlist}}
% \newenvironment{remark}[1][Uwaga]{\begin{trivlist}
% \item[\hskip \labelsep {\bfseries #1}]}{\end{trivlist}}

% definicja czarnego prostokąta zwyczajowo dodawanego na koniec dowodu
\newcommand{\qed}{\nobreak \ifvmode \relax \else
      \ifdim\lastskip<1.5em \hskip-\lastskip
      \hskip1.5em plus0em minus0.5em \fi \nobreak
      \vrule height0.75em width0.5em depth0.25em\fi}

% poniższymi instrukcjami można sterować co ma być numerowane a co nie i co ma być wyświetlane w spisie treści
% \setcounter{secnumdepth}{3}
% \setcounter{tocdepth}{5}

% definicja czcionki mniejszej niż tiny (domyślnie takiej małej nie ma)
\usepackage{lmodern}
\makeatletter
  \newcommand\tinyv{\@setfontsize\tinyv{4pt}{6}}
\makeatother

% definicja jeszcze mniejszej czcionki
\usepackage{lmodern}
\makeatletter
  \newcommand\tinyvv{\@setfontsize\tinyvv{3.5pt}{6}}
\makeatother

% pakiet do obsługi wielostronicowych tabel
\usepackage{longtable}
\setlength{\LTcapwidth}{\textwidth}

\usepackage[section] {placeins}

\usepackage{multirow}

\usepackage{slantsc}
\usepackage[labelsep= space]{caption}
\usepackage[font=small,labelfont=bf]{caption}
\usepackage[justification=centering]{caption}
\addto\captionspolish{\renewcommand{\figurename}{Rys.}}
\addto\captionspolish{\renewcommand{\tablename}{Tab.}}
\addto\captionspolish{\renewcommand*{\appendixpagename}{Dodatki}}
\addto\captionspolish{\renewcommand*{\appendixtocname}{Dodatki}}
\addto\captionspolish{\renewcommand*{\appendixname}{Dodatek}}

%CZCIONKA%
\usepackage{helvet}
\renewcommand{\familydefault}{\sfdefault}
\linespread{1.25}

\usepackage[toc,page]{appendix}
\begin{document}

\chapter{Wprowadzenie teoretyczne} ~\label{sec:chap1} 

\section{ Teoria ewolucji}
Teoria zaprezentowana przez uczonego przyrodnika i geologa Charlesa Darwina w 1859 w pierwszym wydaniu jego książki, “The Origin of Species”,  podsumowującej lata pracy i zebrane doświadczenia na temat rozwoju gatunków. 
W ogólności teoria ewolucji głosi, że wszystkie organizmy żywe są ze sobą spokrewnione i pochodzą od jednego wspólnego przodka. Świat istot żywych podlega ciągłym i stopniowym zmianom, dążącym do adaptacji organizmów, a wszystkie zmiany te są wynikiem doboru naturalnego.   ~\cite{darwin}
\\Teoria Darwina opiera się na pewnym zbiorze zasad  ~\cite{darwin},  ~\cite{darwinWeb}: 
\begin{itemize}
\item{   Prawo zmienności powszechnej i bezkierunkowej}\\
Wyjaśnia, iż jedynie zmienność dziedziczna ma wpływ na ewolucję. Zmienność niedziedziczna, nie wpływa na jej przebieg.
\item{  Prawo różnorodności gatunków}\\ Głosi, że gatunki dzielą się na podgatunki potomne, lub w procesie pączkowania wytwarzają innego rodzaju organizmy potomne.
\item{   Prawo walki o byt}\\ Jest to mechanizm redukujący nadmiar populacji, będący czynnikiem napędzającym proces ewolucji.  Walka o byt może się odbywać między różnymi gatunkami w układzie ofiara - drapieżnik lub w obrębie jednego gatunku w wyniku konkurencji o tę samą niszę ekologiczną. 
\item{    Prawo doboru naturalnego}
\\ Przeżywają  jedynie osobniki najlepiej przystosowane, a formy pośrednie wymierają, co prowadzi do coraz większej rozbieżności cech w następnych pokoleniach i powstania z czasem form bardzo różniących się od pra-przodka i powstawania nowych gatunków.
\item{   Prawo dziedziczenia}
\\ Bezpośrednio łączy się z powyższym prawem. Osobniki słabsze, częściej padające ofiarą, mają mniejsze szanse na rozmnażanie, a co za tym idzie na przekazanie swojego zestawu cech. Powstają organizmy potomne, dziedziczące jedynie cechy od silnych osobników, które przetrwały. 
\end{itemize}
Wszystkie tezy zostały potwierdzone dzięki badaniom z dziedziny biologii molekularnej, ekologii oraz biogeografii. 
  \section{Ewolucja Organiczna}
    Ewolucja to  proces stopniowej przemiany osobników (zarówno zwierząt jak i roślin), który w ostateczności może doprowadzić do powstania nowych gatunków. Przemiana ta może dotyczyć zarówno cech morfologicznych jak i fizjologicznych. Jej istotą jest zmiana składu materiału genetycznego organizmów potomnym w stosunku do organizmów rodzicielskich. ~\cite{encyk}  
Zmiany te mogą być wynikiem różnych mechanizmów:
             \subsection{  Mutacje i zmienność rekombinacyjna }~\label{sec:mutate}
Występowanie tych dwóch mechanizmów ma charakter losowy. Zmienność rekombinacyjna jest wynikiem mieszania się materiału genetycznego, natomiast mutacje spowodowane są zmianami w obrębie jednego organizmu.
    Mutacje być typu punktowego (dotyczące  jednego nukleotydu) lub obejmować większy odcinek DNA (chromosomowe). Wśród nich wyróżniamy:
\begin{itemize}
\item{substytucję, }
\item{delecję,}
\item{insercję,}
\item{tranzycję,}
\item{transwersję}
\item{inwersję, }
\item{deficjencję,}
\item{translokację. ~\cite{berkely}}
\end{itemize}

    Zmienność rekombinacyjna zachodzi dzięki zjawiskom takim jak: crossing over, niezależna segregacja chromosomów i połączenie gamet.
           \subsubsection{ Crossing-over} 
Crossing-over, inaczej krzyżowanie,  jest zjawiskiem wymiany materiału genetycznego między chromatydami 
nie siostrzanymi chromosomów homologicznych podczas procesu mejozy. ~\cite{bioDicWeb} Schemat zachodzenia crossing-over zaprezentowano na grafice rys.~\ref{fig:crossingover}. Chromosomami homologicznymi nazywamy parę chromosomów pochodzących, po jednym, od osobników rodzicielskich. 
\begin{figure}[!h]
		\centering
		\scalebox{.8}{
		\includegraphics[width=0.7\textwidth]{img/crossing_over.jpg}}
		\caption{Profaza mejozy - crossing-over}
		\label{fig:crossingover}
\end{figure}

\subsection{Dobór naturalny}
    To czynnik nadający ewolucji kierunkowy i przystosowawczy charakter. Ma na celu zwiększenie stopnia przystosowania (adaptacji) do warunków środowiskowych 
zarówno na poziomie osobniczym jak i genowym.
 Organizmy posiadające korzystne cechy mają większą szansę na przeżycie 
i rozmnażanie, co prowadzi do zwiększania częstości występowania korzystnych genów 
w populacji.
\subsection{ Dryft genetyczny (zjawisko Wrighta) }
    Dryftem genetycznym nazywa się wahania częstotliwości występowania genu 
nie wynikające z działania doboru naturalnego, migracji, czy mutacji. Jest efektem losowych zmian w ilości alleli w kolejnych pokoleniach. ~\cite{encykBio}
 \subsection{ Hybrydyzacja (krzyżowanie) } ~\label{sec:hybryda}
Hybrydyzacja to proces polegający na krzyżowaniu się osobników, będących przedstawicielami różnych genetycznie populacji, w wyniku którego może powstać potomstwo mieszańcowe. Może to doprowadzić do powstania nowych gatunków, lub przyczynić się do zwiększenia różnorodności genetycznej populacji, bądź pojawienia się w populacji nowych korzystnych cech.~\cite{hybry}
\section{ Strategie ewolucyjne (ES) }
Pojęcie strategii ewolucyjnych powstało w latach pięćdziesiątych XX wieku, gdy naukowcy postawili sobie za cel wykorzystanie teorii ewolucji Darwina  oraz zasady doboru naturalnego na zbiorze potencjalnych wyników do ich optymalizacji.~\cite{javaGen} 
W 1975 roku profesor J.H. Holland jako pierwszy opracował koncept algorytmów genetycznych, które zaprezentowano w książce “Adaption in Natural and Artificial Systems”. Zaproponował on, by zamodelować chromosomy w postaci ciągów zer i jedynek. Tak przygotowany zbiór wejściowy z łatwością ulegać może “ewolucji” poprzez mutację, selekcję, czy też crossing-over.~\cite{javaGen}
Słownik pojęć niezbędnych do poruszania się po temacie zaprezentowano w tabeli~\ref{table:dicTab} \begin{table}
\renewcommand\arraystretch{1.5}
 \centering
    \begin{tabular}{|>{\centering\arraybackslash}m{4cm}|m{8.5cm}|}
     \hline
    \textbf{Pojęcie} & \textbf{Objaśnienie}\\ \hline
     Chromosom& Zakodowana forma potencjalnego rozwiązania zadanego problemu. Ciąg uporządkowanych genów.\\ \hline 
     Gen & Element składowy chromosomu.  \\ \hline
    Osobnik & Dla algorytmów genetycznych, równoważny z pojęciem chromosomu. Niekiedy jednak prezentowany jako zespół chromosomów (genotyp).\\ \hline
     Fenotyp& Odpowiednik genotypu w przestrzeni odkodowanej. \\ \hline
     Populacja& Zbiór osobników o określonej liczebności.\\ \hline
     Przystosowanie& Przystosowanie osobników do zadanego problemu. Oceniane za pomocą funkcji przystosowania. Im większy stopień przystosowania, tym lepsze rozwiązanie. \\ \hline
     Selekcja & Proces filtracji najlepiej dopasowanych osobników z pośród populacji. Wybrane chromosomy trafiają do populacji rodzicielskiej, przygotowywanej do rekombinacji genów.\\ \hline
     Krzyżowanie &  Rekombinacja genów chromosomów rodzicielskich, której wynikiem jest chromosom potomnym o zmienionym składzie. Patrz ~\ref{sec:hybryda}  \\ \hline 
    Rodzic & Chromosom wybrany do krzyżowania. \\ \hline
    Potomek & Wynik krzyżowania pary rodziców. \\ \hline
    Mutacja & Proces zamiany genów w obrębie jednego chromosomu bez wpływu chromosomów rodzicielskich. Patrz ~\ref{sec:mutate}\\ \hline
	\end{tabular}
	 \caption{Słownik pojęć podstawowych} 
    \label{table:dicTab}
\end{table}


    \subsection{ Algorytm ewolucyjny (EA), Algorytm Genetyczny (GA)}
Algorytmem ewolucyjnym nazywamy algorytm probabilistyczny, opierającego się na zasadach obowiązujących w ewolucji organicznej  ~\cite{genAlgWeb}, dla którego generowany jest zbiór osobników $P(t)=\lbrace x_1^t, ..., x_n^t\rbrace$ w każdej iteracji $t$. Każdy osobnik przedstawia potencjalne rozwiązanie zadanego problemu i posiada swoją reprezentację jako struktura danych S. Obiekty zbioru oceniane są w oparciu o ich “dopasowanie”. W iteracji $t+1$ tworzy się nową populację osobników. Jest ona wynikiem selekcji najlepiej “dopasowanych” obiektów z iteracji $t$. Niektóre z wybranych podlegają transformacji (mutacja / crossing-over) dając nowe rozwiązania. Po zakończeniu działania algorytmu oczekuje się, iż najlepsze możliwe osobniki znajdą się w zbiorze końcowym i reprezentują rozwiązanie znajdujące się blisko optymalnego (rozwiązanie rozsądne).[ ~\cite{algBook}] W ten sposób unika się przeszukiwania całej przestrzeni w poszukiwaniu rozwiązania, a jedynie wybierana zostaje niewielka populacja jej przedstawicieli. A dzięki mutacjom otrzymuje się rozwiązania coraz lepsze, bliskie optimum.  Ogólny schemat blokowy działania algorytmu przedstawiono na rysunku ~\ref{fig:algSchem}.

\begin{figure}[!h]
		\centering
		\scalebox{.8}{
		\includegraphics[width=0.7\textwidth]{img/program_ewolucyjny_alg.jpg}}
		\caption{Struktura programu ewolucyjnego}
		\label{fig:algSchem}
\end{figure}

Podczas  analizy literatury stwierdzono, iż nazwy “algorytm ewolucyjny”  oraz “algorytm  genetyczny” stosuje się zamiennie i  poniższym tekście również przyjęto taką koncepcję. 

\subsection{ Algorytm genetyczny a program ewolucyjny}
 Na podstawie algorytmów ewolucyjnych powstały programy ewolucyjne. Ich struktura pozostaje taka sama, jednak różnice widać na niższym poziomie. 
Dla algorytmów przyjęto zapis w postaci skończonego, uporządkowanego ciągu  jasno zdefiniowanych czynności, koniecznych do wykonania pewnego zadania. Konieczny do rozszyfrowania tego zapisu jest specjalny parser, który zamienia ciąg w wykonalną funkcję oraz rozpoznaje ewentualne zmiany stanu (wywołane mutacją, bądź crossing-over), które mogłyby zagrażać jego działaniu.  W porównaniu do tego program ewolucyjny jest przedstawiony jako struktura drzewiasta czynności i wartości. Również niezbędny jest parser, jednak pomniejszony o świadomość stanów ( te ukryte są wewnątrz struktury). 
\\
Poza tym znaczącą różnicę stanowi reprezentacja chromosomów. Dla algorytmów ewolucyjnych/genetycznych chromosomy muszą być w formie binarnej, natomiast program pozwala nam na zdefiniowanie dowolnych struktur.
Związane z tym jest również zapotrzebowanie na wprowadzenie spersonalizowanych operatorów genetycznych, odpowiednich dla zadanej struktury i zadania, podczas gdy algorytmy korzystają z podstawowych operatorów.
\\
Algorytmy genetyczne wymagają modyfikacji zadania (przetworzenie na łańcuch binarny). Nie jest to zadaniem łatwym i niekiedy może wymagać użycia parserów, czy też algorytmów naprawy. Np. Reprezentacja indeksów liczby z zakresu od 1 - 5, możliwa jest dzięki 3 bitom. Jednak podczas procesu mutacji mogą powstać indeksy wykraczające poza zakres (6-8). Zmienienie ich wartości do zgodnych z zakresem wymaga użycia specjalnego algorytmu naprawy. 
Programy ewolucyjne, w odróżnieniu,  wymagają zmiany reprezentacji chromosomowej potencjalnych rozwiązań oraz wytworzenia odpowiednich operatorów genetycznych do działania na wytworzonych strukturach.  Zależności te w sposób schematyczny przedstawiono na rysunkach rys.~\ref{fig:roznice_alg}, rys.~\ref{fig:roznice_prog}.
\begin{figure}[!h]
		\centering
		\scalebox{.8}{
		\includegraphics[width=0.7\textwidth]{img/schemat_alg_gen.jpg}}
		\caption{Schemat działania algorytmu genetycznego.}
		\label{fig:roznice_alg}
		\scalebox{.8}{
		\includegraphics[width=0.7\textwidth]{img/schemat_prog_ew.jpg}}
		\caption{Schemat działania programu ewolucyjnego.}
		\label{fig:roznice_prog}
\end{figure}
\subsection{Wymagania}
Zarówno program jak i algorytm posiadają listę wymagań, które muszą zostać spełnione, by zapewnić ich poprawne działanie.~\cite{algBook}\\
Musi istnieć:
~\begin{itemize}
\item
zbiór z reprezentacją możliwych rozwiązań problemu,
\item
metoda generowania początkowej populacji potencjalnych rozwiązań, 
\item
funkcja oceniająca, do oceny “dopasowania” rozwiązań,
\item
operator “genetyczny”, wpływający na populację,
\item
parametr populacji niezbędny algorytmowi (np. rozmiar populacji, prawdopodobieństwo mutacji, długość wykonywania się algorytmu itp.) 

\end{itemize}




   \end{document} 